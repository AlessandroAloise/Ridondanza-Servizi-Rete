\documentclass[../main.tex]{subfiles}
\begin{document}


\subsection{Informazioni sul progetto}

  \begin{itemize}
    \item \textbf{Sezione}: Informatica
    \item \textbf{Classe:} I4AC
    \item \textbf{Supervisore:} Pascal Poncini
    \item \textbf{Title:} Ridondanza Servizi Rete
    \item \textbf{Data Inzio}: 2022-09-29
    \item \textbf{Data Fine}: 2022-12-07
\end{itemize}


\begin{itemize}
    \item \textbf{Documentazione}:  una documentazione completa del lavoro svolto
    \item \textbf{Diari}: Aggiornamenti costante per ogni sessione di lavoro
\end{itemize}

\subsection{Abstract}
\textit{Every company will have experienced at least once when one of its servers stops working. Today the technology is in constant development, more than ever the up time of servers is important, many companies have equipped themselves by having redundancies of duplicate servers to be always operational, but this will be heavy for memory because of duplicate servers that they're useless. This project aims at an optimization of servers utilisation thus removing  the duplication of unnecessary services through new technologies. This is intended to be an alternative and a set of current of aviable solutions so they can solve the problem that many companies have}

\subsection{Scopo}

L'idea di questo progetto è di creare una ridondanza dei servizi che saranno attivi su entrambi i server. In questo progetto si dovranno poter gestire tutti quanti i servizi necessari (ADDS,DNS,GPO,DHCP,File server), inoltre questi server saranno anche in grado di gestire tutte le cartelle di rete. Nel progetto si vogliono eliminare tutti i doppioni presenti nei vari server che sono inutilizzati o non necessari, perciò essi non dovranno essere presenti. Si vogliono porer usare tutte le potenzialità utilizzabili e tool di windows server 2019, al fine di tenere un server pulito e ordinato senza avere dei doppioni che occupano solamente spazio e creano casino che potrebbe essere utilizzato per altri scopi
\end{document}  
